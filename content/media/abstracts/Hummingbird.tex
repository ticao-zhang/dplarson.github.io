\documentclass[10pt]{llncs}

% adjust page size
\setlength{\hoffset}{0.0pt}             % default value 0.0pt
\setlength{\voffset}{0.0pt}             % default value 0.0pt
\setlength{\evensidemargin}{0.25in} 	% default value 64.0pt
\setlength{\oddsidemargin}{0.15in} 	    % default value 64.0pt
\setlength{\topmargin}{-46.94011pt}     % default value 22.0pt
\setlength{\headheight}{72.0pt}         % default value 12.0pt
\setlength{\headsep}{18.06749pt}        % default value 18.06749pt
\setlength{\textheight}{600.47656pt}    % default value 550.0pt
\setlength{\textwidth}{6.00in} 		    % default value 341.0pt
\setlength{\marginparsep}{7.0pt}        % default value 7.0pt
\setlength{\marginparwidth}{128.0pt}    % default value 71.0pt
\setlength{\marginparpush}{5.0pt}       % default value 5.0pt
\setlength{\footskip}{25.29494pt}       % default value 25.29494pt


\title{The Effects of Turbulent Flow on Hummingbird Kinematics and Metabolism}

\author{
    David P. Larson
    \and 
    Nir Sapir
    \and 
    Robert Dudley
}

\institute{
    \textit{Department of Integrative Biology, University of California, Berkeley}
    \\[0.25cm]
    Summer 2011
}

\begin{document}

\maketitle

\begin{abstract}
This study investigates the effects of air turbulence on the flight kinematics and metabolism of Anna’s hummingbirds (Calypte anna). Turbulent air flow is generated and controlled through the use of static mesh grids and an active array of fans. A Parti- cle Image Velocimtery (PIV) system, consisting of a high speed camera (500 frames per second) and a Nd:YAG laser, were utilized to measure the flow. The flow’s Turbulent Kinetic Energy (TKE) and integral length scale were calculated from the PIV data using autocorrelation and autocovariance analysis, thereby quantifying the flow’s level of turbulence. Variations in wing and body kinematics were analyzed using multiple angle, high speed video while the metabolic rate was measured by the bird’s volumetric oxygen consumption rate during flight. Preliminary results did not show a conclusive correlation between the produced turbulent conditions and variations in the flight kine- matics or metabolism of the tested hummingbirds. Future work will focus on expanding the range of tested turbulent conditions.
\end{abstract}


\end{document}
