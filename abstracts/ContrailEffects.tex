\documentclass[10pt]{llncs}

% adjust page size
\setlength{\hoffset}{0.0pt}             % default value 0.0pt
\setlength{\voffset}{0.0pt}             % default value 0.0pt
\setlength{\evensidemargin}{0.25in} 	% default value 64.0pt
\setlength{\oddsidemargin}{0.15in} 	    % default value 64.0pt
\setlength{\topmargin}{-46.94011pt}     % default value 22.0pt
\setlength{\headheight}{72.0pt}         % default value 12.0pt
\setlength{\headsep}{18.06749pt}        % default value 18.06749pt
\setlength{\textheight}{600.47656pt}    % default value 550.0pt
\setlength{\textwidth}{6.00in} 		    % default value 341.0pt
\setlength{\marginparsep}{7.0pt}        % default value 7.0pt
\setlength{\marginparwidth}{128.0pt}    % default value 71.0pt
\setlength{\marginparpush}{5.0pt}       % default value 5.0pt
\setlength{\footskip}{25.29494pt}       % default value 25.29494pt


\title{Contrail Effects On Ground-Based Solar Irradiance}

\author{
    Marina Fernandez 
    \and 
    David P. Larson 
    \and 
    C.F.M. Coimbra
}

\institute{
    \textit{Mechanical and Aerospace Engineering, University of California, San Diego}
    %\\[0.25cm]
    %Fall 2012
}

\begin{document}

\maketitle

\begin{abstract}
The impact of aircraft condensation trails (contrails) on ground-based solar irradiance measurements, and therefore solar power plant output, has yet to be formally analyzed for use in solar resourcing and integration. This work quantifies the effects of persistent contrails on Direct Normal Irradiance (DNI) using high-fidelity solar irradiance measurements and sky imaging. Contrails were identified using sky images from the Coimbra Solar Forecast Engine Lab's observatory at the University of California, San Diego and then correlated to intra-minute, ground-based irradiance measurements. Selecting contrails that specifically blocked the sun allowed us to observe their effect on ground irradiance, specifically on DNI data, as opposed to more general effects on the global and diffuse irradiance. The contrail-correlated solar irradiance data is analyzed by calculating the magnitude of the irradiance drop for each contrail within specified sun-to-contrail configurations on a clear day. Results showed that contrail dissipation rates are significant factors in understanding the effects of contrails in general, with dissipative contrails affecting DNI in a similar extent to that of cloud cover over a multi-minute time interval and non-dissipative contrails producing sharp drops in DNI over time intervals of two minutes or less. On a clear day, dissipative and non-dissipative contrails can cause drops in DNI of greater than 500 W/m$^2$ and 150 W/m$^2$ respectively. These results show that that contrails should be considered in both solar power plant site selection and power output forecasting.
\end{abstract}


\end{document}
